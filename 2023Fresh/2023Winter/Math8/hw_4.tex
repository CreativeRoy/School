\documentclass[12pt]{article}

\usepackage[a4paper, total={6in, 8in}]{geometry}
\usepackage{enumitem}
\usepackage{amssymb}
\usepackage{amsmath}
\usepackage{array}
\usepackage{fancyhdr}
\renewcommand{\headrulewidth}{0pt}

\pagestyle{fancy}

\setlength{\headheight}{35pt} 
\lhead{Math 8 \\ Homework 4} % Left header is empty
\rhead{Roy Zhou \\ \today} % Right header contains date and name

\geometry{margin=1in}
\noindent 

\begin{document}

\section*{Ch. 4}

\subsection*{Exc. 2}
We must prove that if x is an odd integer, than $x^3$ is odd. We can start off by
utilizing our given to write $x$, as $2n+1$, where $n \in \mathbb{Z}$. Then we can 
cube $2n+1$, this yields $8n^3 + 12n^2 + 6n + 1$. Factoring out all factors of 2,
we can write the expression in the form of $2(m) + 1$, where $m$ is an integer because
it is equal to $4n^3 + 6n^2 + 3n$. In the end, we get that $x^3$ is equal to a number of 
the form $2m+1$, with $m$ being an integer. By definition, $2m+1$ must be odd, so $x^3$ must be odd.

\subsection*{Exc. 8}
We are given the facts that $a$ is an integer, as well as the fact that $5|2a$. We can use
these givens to write that $2a = 5b$ for $b \in \mathbb{Z}$. Further because there is a 
2 on the side of $2a$, we know that both numbers must be even. That means that we can write 
$b$ as $2c$, where $c \in \mathbb{Z}$. We can plug it all in to get that $2a = 5(2c)$, simplifying,
we get $a = 5c$. Now we can plug $5c$ into the end, and because $c$ is an integer, we know that
the statement $5|5c$ is true.

\subsection*{Exc. 14}
Any integer is either even or odd, so intuitively, we can try $n$ being even or odd to prove that
$5n^2 + 3n + 7$ is odd. Lets assume $n$ is odd, then we can write $n = 2m+1$, where $m \in \mathbb{Z}$.
Plugging $2m+1$ into the expression, we get: 
$$5(2m+1)^2 + 3(2m+1) + 7$$\\
$$20m^2 + 26m + 15$$ \\
Now again, we can factor the equation such that $p = 10m^2 + 13m + 7$. We can then write
the expression as $2p + 1$, with $p$ being an integer, proving that the expression is odd.
To test the even case, we can write $n = 2m$. Now, we get: 
$$20m^2+6m+7$$
which can be written as $2p + 1$, where $p = 10m^2 + 3m + 3$. We have proved that both cases,
$n$ either being even or odd, still result in the expression being odd.

\subsection*{Exc. 16}
We look at the cases of if both integers are odd, we get that $$2m+1 + 2n+1 = 2(m+n+1)$$. 
Clearly this is even, as $m$ and $n$ are both integers, meaning that $m+n+1 \in \mathbb{Z}$.
Furthermore, $2$ times an integer is an integer. \\
We look at the case of if both integers are even, we get that $2m + 2n = 2(m+n)$. Using the same logic as above,
we know that $(m+n)$ is an integer, thus the sum of two even integers is even. Thus by doing all cases,
we have proved that if two integers with same parity sum together, we get an even integer.

\subsection*{Exc. 26}
We can prove that every odd integer is the difference of two squares constructively. First look at what happens
at the difference of two consecutive squares(let $m$ be an integer): 
$$(m+1)^2 - (m)^2 = (m+1 + m)(m+1-m) = (2m+1)(1)$$\\
Coincidentally, $2m+1$ is also the form of every odd integer. This means that to get any given odd integer,
one must split the integer into a sum of two consecutive numbers, then the difference between the squares of the two consecutive numbers
will be the desired odd number. 

\section*{Ch. 5}

\subsection*{Exc. 6}
Suppose that $x \le -1$ we want to show that it implies $x^3 - x \le 0$. We can show this by
factoring the expression: 
$$x(x^2-1)$$
This seperates the problem into two different parts, since $x \le -1$, that means that the $x$ term
by itself is always going to be negative. On the other hand, the $x^2 - 1$ term is always going to be non-negative.
It will be 0 in the case that $x = -1$, and positive in any other case. Thus we have it that $x^3 - x$ is always either zero,
or negative. Thus by contrapositive, we are able to prove that $x^3 - x > 0$ implies 
$x > -1$.

\subsection*{Exc. 10}
Suppose that $x|y$ or $x|z$. WLOG let's assume that $x|y$ we can then write $y = ax$. Then,
we can substitute $ax$ into the end expression, getting that $x|(ax)(z)$. Because there is an $x$ in the 
numerator, we know that $x|azx$ is true. Thus we have proved the statement through contrapositive.

\subsection*{Exc. 12}
Supposse $a$ is even. We can write $a = 2b$, where $b \in \mathbb{Z}$. We can then express $a^2$
as $4b^2$. Clearly $4|4b^2$ is true, as $b$ is an integer, and there is a 4 in the numerator.
Thus we have proved through contrapositive that if $a^2$ is not divisible by 4, then a is odd.

\subsection*{Exc. 18}
We can expand $(a+b)^3$ into $a^3 + 3a^{2}b+3ab^2+b^3$ when we mod the expression by $3$, we are able 
to remove any terms that are a multiple of $3$, this leads us with $a^3 + b^3$. Thus we have proved that
$a^3 + 3a^{2}b+3ab^2+b^3 \equiv a^3 + b^3$ mod 3.

\subsection*{Exc. 24}
We can write $a \equiv b \pmod{n}$ as $a = b + xn$ and $c \equiv d \pmod{n}$ as $c = d + yn$, where $x,y \in \mathbb{Z}$.
We can then multiply the two equations together, yielding:
$$ac = bd + byn + dxn + xyn^2$$ \\
Modding by n, we get that:
$$ac \equiv bd \pmod{n}$$
This happens as all terms with n are effectively cancelled out. Now, we have proved that if
$a \equiv b \pmod{n}$ and $c \equiv d \pmod{n}$ implies that $ac \equiv bd \pmod{n}$.

\section*{Ch. 6}

\subsection*{Exc. 6}
Proof by contradiction. Suppose that $a, b \in \mathbb{Z}$, then we want to see if $a^2-4b-2 = 0$.
To prove that this is impossible, we can first rearrange to: 
$$a^2 = 4b+2$$\\
From here, we know that since the right side has a factor of 2, that $a$ must be even. In other words,
$a = 2c$ for $c$ is an integer. We can plug in to get $4c^2 = 4b + 2$. However, now we have a contradiction,
as the right side must be divisible by 4. We get $c^2 = b + \frac{2}{4}$. Rearranging again, we get that 
$$c^2 - b = \frac{2}{4}$$ \\
This is clearly a contradiction, as the difference between two integers cannot be a fraction, thus 
we know that the if $a, b \in \mathbb{Z}$, then $a^2 -4b -0 \ne 0$.
\end{document}


