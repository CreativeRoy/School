\documentclass[12pt]{article}

\usepackage[a4paper, total={6in, 8in}]{geometry}
\usepackage{enumitem}
\usepackage{amssymb}
\usepackage{amsmath}
\usepackage{array}
\usepackage{fancyhdr}
\renewcommand{\headrulewidth}{0pt}
\pagestyle{fancy}
\geometry{margin=1in} 
\geometry{bottom=1.5in}
\setlength{\headheight}{35pt} 

\lhead{Math 8 \\ Homework 8} % Left header is empty
\rhead{Roy Zhou \\ \today} % Right header contains date and name

\begin{document}

\begin{center} 
\Large
\textbf{Equivalence Relations and Surjective-Injective Functions}
\end{center}

\noindent
\textbf{Problem 4:} \\

\noindent
We get two tables, as labeled:

\begin{center}	
\begin{tabular}{c| c c c c c c}
 	$\times$  & [0] & [1] &[2]  & [3] & [4] & [5] \\
	
	\hline
	[0] & [0]& [0]& [0]& [0]& [0]& [0]\\

	[1] & [0]& [1]& [2]& [3]& [4]& [5]\\
	
	[2] & [0]& [2]& [4]& [0]& [2]& [4]\\
	
	[3] & [0]& [3]& [0]& [3]& [0]& [3]\\
	
	[4] & [0]& [4]& [2]& [0]& [4]& [2]\\
	
	[5] & [0]& [5]& [4]& [3]& [2]& [1] 
\end{tabular}
\end{center}

\begin{center}	
\begin{tabular}{c| c c c c c c}
 	$+$  & [0] & [1] &[2]  & [3] & [4] & [5] \\
	
	\hline
	[0] & [0]& [1]& [2]& [3]& [4]& [5]\\

	[1] & [1]& [2]& [3]& [4]& [5]& [0]\\
	
	[2] & [2]& [3]& [4]& [5]& [0]& [1]\\
	
	[3] & [3]& [4]& [5]& [0]& [1]& [2]\\
	
	[4] & [4]& [5]& [0]& [1]& [2]& [3]\\
	
	[5] & [5]& [0]& [1]& [2]& [3]& [4] 
\end{tabular}
\end{center} \\

\noindent
\textbf{Problem 6: }\\

\noindent
It is \textbf{not} true that necessarily $[a]=[0]$ or $[b]=[0]$. This is because $6 = 2 \cdot 3$,
so we can have $[a] = [2], [b] = [3]$, and that would also get  $[a]\cdot[b] = [0]$. This, however, 
is true for $\mathbb{Z}_7$, as $7$ is a prime number, meaning that for two numbers to multiply to 
equal $7$, one has to be a factor of $7$, hence it must be in the equality class of $[0]$. \\

\noindent
\textbf{Problem 10: }\\
	
\noindent
We can prove bijectiveness by proving injectiveness, than surjectiveness. To prove 
injectiveness, we use the contrapositive approacj, we suppose that $f(a)=f(b)$ for $a,b \in \mathbb{Z}-\{1\}$, 
and we aim to prove that $a = b$. This can be done by first writing out the equation:
\begin{center}
	$$f(a) = f(b) $$
	$$\left(\frac{a+1}{a-1}\right)^3 = \left(\frac{b+1}{b-1}\right)^3$$
	$$\frac{a+1}{a-1} = \frac{b+1}{b-1}$$
	$$ab-a+b-1 = ab-b+a-1$$
	$$2b = 2a$$
	$$a=b$$
\end{center}
We have proven $a = b$, thus we are done with the injectiveness. To prove surjectiveness, we just
suppose there is  $b \in \mathbb{Z}-\{1\}$, and we try to prove there exists  $a \in \mathbb{Z}-\{1\}$,
such that $f(a) = b$. To do this, we start off with 
\[
\left(\frac{a+1}{a-1}\right)^3 = b
.\]
Now, we solve for $a$, and we need to make sure that not only does $a$ exist, but that its also in 
the co-domain.
\begin{center}
	$$\left(\frac{a+1}{a-1}\right)^3 = b$$
	$$\frac{a+1}{a-1} = b^{\frac{1}{3}}$$
	$$a+1=b^{\frac{1}{3}}(a-1)$$
	$$a-ab^{\frac{1}{3}} = -1-b^{\frac{1}{3}}$$ 
	$$a(b^{\frac{1}{3}}-1) = 1+b^{\frac{1}{3}}$$
	$$a = \frac{1+b^{\frac{1}{3}}}{b^{\frac{1}{3}}-1}$$
\end{center}
Because $b \neq 1$, we get that there is always going to be a real, $a$.
Now to prove that  $a \neq 1$, we can put set $a = 1$ and prove by counterexample.
\begin{center}
	$$1 = \frac{1+b^{\frac{1}{3}}}{b^{\frac{1}{3}}-1}$$
	$$1+b^{\frac{1}{3}} = b^{\frac{1}{3}}-1$$
	$$1 = -1$$
\end{center}
Clearly this is not true, so our proof of bijection is done!\\

\noindent
\textbf{Problem 12: }\\

\noindent
We consider surjectiveness first, we are able to reach all positive integers by setting
$a = 0$ and just having $b$ be the integer. In equation, we get: 
\[
\theta(0, b) = b  
.\] 
If our number is zero or negative, then we set $x = 1$ and get:
\[
\theta(1, b) = 1-b
.\] 
This way, we just set $b$ to be one greater than the absolute value of the non-positive number. 
We have proved surjectivness, now to prove injectiveness, we use the direct method, supposing that
$a_1,a_2 \in \{0, 1\}$ and  $b_1,b_2 \in \mathbb{N}$, and $a_1\neq a_2$, $b_1 \neq b_2$. Now we need to prove
that $\theta(a_1,b_1) \neq \theta(a_2,b_2)$. This is relatively simple, as WLOG we can set $a_1 = 0, a_2 = 1$.
We then get:
\begin{center}
	$$\theta(a_1,b_1) = b_1$$
	$$\theta(a_2,b_2) = 1 - b_2$$
\end{center} 
If $\theta(a_1,b_1) = \theta(a_2,b_2)$, it would imply:
\begin{center}
	$$b_1=1-b_2$$
	$$b_1+b_2=1$$
\end{center}
It is impossible for $b_1 + b_2$ to equal $1$, as both $b_1, b_2 \in \mathbb{N}$, meaning that their 
respective minimum value is $1$, which means that their sum is at least $2$. Thus we have proved surjectiveness
and injectiveness. Which means that the thing is bijective. \\

\noindent
\textbf{Problem 2: } \\

\noindent
First, we can think about what it means for $a^{k}-a^{\ell}$ to be divisible by $10$. It would mean that the 
units digits of $a^{k}, a^{\ell}$ are the same. The set of possible units digits can be written as 
$S = \{0, 1, 2, \ldots, 8, 9\}$, there are only 10 members of the $S$. However, there are infinitely(more than 10)
members of the natural numbers, meaning that the mapping from $\mathbb{N} \to S$ is not injective,
and for some $k, \ell \in \mathbb{N}$, the units digit of $a^{k}, a^{\ell}$ are the same,
which means that their difference will be divisible by 10.

\end{document}
