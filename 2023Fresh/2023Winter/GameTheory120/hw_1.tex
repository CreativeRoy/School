\documentclass[12pt]{article}

\usepackage[a4paper, total={6in, 8in}]{geometry}
\usepackage{enumitem}
\usepackage{amssymb}
\usepackage{amsmath}
\usepackage{array}
\usepackage{fancyhdr}
\renewcommand{\headrulewidth}{0pt}

\pagestyle{fancy}

\setlength{\headheight}{35pt} 
\lhead{Game Theory 120 \\ Homework 1} % Left header is empty
\rhead{Roy Zhou \\ \today} % Right header contains date and name

\geometry{margin=1in} 

\begin{document}

\noindent
\textbf{Problem 1:}\\

We can do this problem inductively. First consider the case in which $m = 0$ and $n = 1$.
We get that player 2 wins, as player 1 is forced to eat the only block, which happens to be poisoned. 
We can then look at the case in which $m = 1$ and $n = 2$. We see that again, player 2 wins, as player 1 can only eat
one of the blocks on the edges. The player 2 can then eat the only remaining unpoisoned cube. Then the case degenerates to $m = 0$ and $n = 1$.
We also observe that from any case in which $n = m + 1$, player 2 will always win, as they can manipulate it to
be a situation where $n = m + 1$ on player 1's turn. Eventually $n,m$ will keep decreasing, until player 1 will have to eat the poisoned cube. 
On the other hand, for any case in which $n \ne m+1$, player 1 will win, as they can constantly adjust the situation so that player 2 plays 
on a gamestate where $n = m+1$. For the same reason, player 2 will eventually lose. Thus we have
concluded that player 1 wins when $n \ne m+1$ and player 2 wins otherwise.\\

\noindent 
\textbf{Problem 2:}\\

We observe that symmetry plays a big role in this game. In most cases, if one player can mirror the actions of the other,
the player mimicking will win. We also observe that mirroring does not work if we get into a situation where there are two pins left
that are adjacent to each other(they have to be adjacent at the center, or else the person mirroring wasn't actually mirroring).
Because of this, Player 1 will always have a winning strategy, because they can split the center and create two equally sized sections. From
there, they can just mirror the actions of Player 2. 

\noindent
\textbf{Problem 3: }\\

(This is assuming we are not playing with the Pie rule, in which player 2 can swap with palyer 1)\\
We can try a few easier cases, starting off small at $n = 1,2,3$. We get that player 1 wins, simply through the fact that 
they are able to make it across in $n$ without much trouble. At slightly higher values for $n = 4, 5, 6$ we still get that the first player wins.
Without many cases, we are very quickly able to land on the (apparently right) assumption that player one is favored to win.
While an exact reasoning might be a bit far fetched, one easy way to think of it, is that unlike in a game of chess or checkers, a move is never
non-beneficial. So the first player getting effectively one more move must be beneficial.

\end{document}


