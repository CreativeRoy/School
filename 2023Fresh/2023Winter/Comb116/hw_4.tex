\documentclass[12pt]{article}

\usepackage[a4paper, total={6in, 8in}]{geometry}
\usepackage{enumitem}
\usepackage{amssymb}
\usepackage{amsmath}
\usepackage{array}
\usepackage{fancyhdr}
\renewcommand{\headrulewidth}{0pt}
\pagestyle{fancy}
\setlength{\headheight}{35pt} 
\geometry{margin=1in} 

\lhead{Game Theory 120 \\ Homework 4} % Left header is empty
\rhead{Roy Zhou \\ \today} % Right header contains date and name

\begin{document}

\noindent
\textbf{Problem 6: }\\

Starting with the obvious, the coefficient of $x^{8}y^{9}$ is equal to $0$,
as only terms with powers that sum to $18$ are represented in the expansion. 
The coefficient of $x^{5}y^{13}$ is  $\binom{18}{5}(3^{5})(-2^{13})$.\\

\noindent
\textbf{Problem 8: }\\

We can write $2^{n} = (3-1)^{n}$. As shown through the binomial theorem 
\[
	 2^n = (3-1)^{n} = \Sigma^{n}_{k=0}\binom{n}{k}3^{n-k}(-1)^{k} 
.\] 
As we can see, there isn't much left to do, as the right side of our expansion is exactly 
the right side of the given equation, thus we have shown how the binomial theorem leads to the expression. 
\\

\noindent
\textbf{Problem 16: } \\

We know that 
\[
	(1+x)^{n} = \sum^{n}_{k=0}\binom{n}{k}x^{k}
.\] 

Thus, we can take the integral from $[0, 1]$ on both sides. For the right
 \[
	 \int_{0}^{1} (1+x)^{n}dx = \frac{1}{n+1}(1+x)^{n+1}|^{1}_{0}
	 = \frac{2^{n+1} - 1}{n+1}
.\] 
This happens to be identical to the right side of the equation. Now for the left:
\[
	\sum^{n}_{k=0}\binom{n}{k}\int^{1}_{0}x^{k} = \sum^{n}_{k=0}\binom{n}{k}\frac{1}{k+1} \\
	= 1 + \frac{1}{2}\binom{n}{1} + \frac{1}{3}\binom{n}{2} + \cdots + \frac{1}{n+1}\binom{n}{n}
.\] 
We started off with a basic binomial expansion, which we know is true based off the binomial theorem.
Through integrating it, we are able to generalize a proof, for every $n$.\\

\noindent
\textbf{Problem 20: } \\

The phrasing of the problem makes it inherently easy. Since we know that $\binom{n}{m} = 0$ when $n < m$, we can solve by trying $m = 1, 2, 3$
in that order. We get: 
\begin{align}
	1^3 = 0 + 0 + c \\
	2^3 = 0 + b + 2c\\
	3^3 = a + 3b + 3c\\
\end{align}

Using some simple algebra, we get $c = 1, b = 6, a = 6$. Thus we can generalize for all integers, 
\[
	m^3 = 6\binom{m}{3} + 6\binom{m}{2} + \binom{m}{1} 
.\] 
To evaluate the sum of $1^3 + 2^3 + 3^3 + \cdots + n^3$ We first expand all cubes to the previously
proven form: 
\[
	1^3 + 2^3 + 3^3 + \cdots + n^3 = 6\binom{1}{3} + 6\binom{1}{2} + \binom{1}{1} + 6\binom{2}{3} + 6\binom{2}{2} + \binom{2}{1} + \cdots + 6\binom{n}{3} + 6\binom{n}{2} + \binom{n}{1}
\]
\[
	= 6(\binom{3}{3} + \binom{4}{3} + \cdots + \binom{n}{3}) + 6(\binom{2}{2} + \binom{3}{2} + \cdots + \binom{n}{2})) + (1 + 2 + 3 + \cdots + n)		
\]
\[
	= \frac{n(n+1)}{2} + 6(\binom{3}{3} + \binom{4}{3} + \binom{5}{3} + \cdots + \binom{n}{3}) + 6(\binom{2}{2} + \binom{3}{2} + \binom{4}{2} + \binom{5}{2} + \cdots + \binom{n}{2})

\] 
\[
	= \frac{n(n+1)}{2} + 6(\binom{n+1}{4}) + 6(\binom{n+1}{3})
\]
\[
	= \frac{1}{4}(n+2)(n+1)(n)(n-1) + \frac{1}{4}(n+1)(n)(n-1) + \frac{n(n+1)}{2}
\] 
\[
	= \frac{n^{2}(n+1)^{2}}{4}
.\] \\

\noindent
\textbf{Problem 28: }\\

We can manipulate the equation to be:	
$$\frac{\sum^{n}_{k=1}k\binom{n}{k}\binom{n}{n-k}}{n} = \binom{2n-1}{n-1}$$
$$\sum^{n}_{k=1}\binom{n-1}{k-1}\binom{n}{n-k} = \binom{2n-1}{n-1}$$
Now we can use an elementary combinatorics argument, say we want to find a group of $n-1$ people in a group of $2n-1$ people.
We can first split it them two groups $A, B$, with $A$ having $n-1$ people and $B$ having $n$ people. Then we can 
select $k-1$ people from $A$ and $n-k$ people from $B$ getting us $\binom{2n-1}{n-1}$.\\

\noindent 
\textbf{Problem 47: } \\

$$10^{\frac{1}{3}} = 2(1 + \frac{1}{4})^{\frac{1}{3}} = 2\sum^{\infty}_{k=0} \binom{\frac{1}{3}}{k}(\frac{1}{4})^{k}$$
$$= 2(1 + \frac{1}{4}(\frac{1}{3}-1) + (\frac{1}{4}^{2}) \frac{(\frac{1}{3})(\frac{1}{3}-1)}{2} + (\frac{1}{4}^{3}) \frac{(\frac{1}{3})(\frac{1}{3}-1)(\frac{1}{3}-2)}{6}) + \cdots$$
$$= 2.1527$$

\noindent	
\textbf{problem 60: }

Given $x \in \mathbb{Z}$ 


\end{document}
