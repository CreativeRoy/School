\documentclass[12pt]{article}

\usepackage[a4paper, total={6in, 8in}]{geometry}
\usepackage{enumitem}
\usepackage{amssymb}
\usepackage{amsmath}
\usepackage{array}
\usepackage{fancyhdr}
\renewcommand{\headrulewidth}{0pt}

\pagestyle{fancy}

\setlength{\headheight}{35pt} 
\lhead{Comb 116 \\ Homework 2} % Left header is empty
\rhead{Roy Zhou \\ \today} % Right header contains date and name

\geometry{margin=1in}

\begin{document}

\noindent
\textbf{Problem 8}\\
We can approach this problem by analyzing the division algorithm. We can quickly calculate the 
integer portion of the decimal expansion through the equation $m = p_{0}n + r_{0}$. Here, $p_0$
is our integer portion of the decimal expansion. We are left to work with $r_0$, which is 
$0 \le r_0 \le n-1$. From here, we can calculate each decimal digit by finding $p_i$ where $i > 0$.
	We are able to do this, as because $r$ is less than $n-1$, we know that $p$ must be a single digit.
$$10r_{0} = p_{1}n + r_{1}$$
$$10r_1 = p_{2}n + r_{2}$$
$$10r_2 = p_{3}n + r_3$$
$$\cdots$$
$$10r_9 = p_{10}n + r_{10}$$
$$10r_{10} = p_{11}n + r_{11}$$
We know that there are only $10$ possible integer values for $r_i$, that is the single digits of
$0, 1, 2, \cdots, 9$. Because $i$ goes to infinite(or in this example 11), that means that there are some $r_i, r_j$
which are bound to be the same. This means that the resulting $r_i+1 = r_j+1$, and thus we have an inductive pattern, as
every $r_i$ leads to exactly one $r_{i+1}$.

\noindent 
\textbf{Problem 12} \\
The Chinese Remainder Theorem states the existence of some $x$ that satisfies the following conditions:
$$x \equiv a_1 \pmod{b_1}$$
$$x \equiv a_2 \pmod{b_2}$$
Where $b_1$ and $b_2$ are coprime. However, there are instances where the Chinese Remainder Theorem works despite $b_1$
and $b_2$ being non-coprime. This is when $a_1 - a_2|\gcd{(b_1, b_2)}$. We don't need to prove this, and only need
to show that it is possible, by example. Given the conditions where:
$$x \equiv 2 \pmod{4}$$
$$x \equiv 4 \pmod{6}$$
There are many solutions, but the smallest one is when $x = 10$. Thus we have shown that it is possible
for the Chinese Remainder Theorem to work, despite not being coprime.

\noindent
\textbf{Problem 26}\\
Consider a person, $P$, in row $j$ at column $k$ of the sorted list. In the original, only row-sorted list, 
we know that there is one person to the left of $P$ who is shorter. Further, we know that $P$ is taller than all the people in front
in his own column. Those respective people in the same column are also taller than at least $1$ person in row $k-1$. Thus we know that there
are exactly $j$ people who are shorter than $P$, however, since $P$ is on the $jth$ row, that means that his counterpart in row $k-1$ will be 
shorter.

\noindent
\textbf{Problem 29}\\
We can look at this problem through a purely pigeon-hole based standpoint. Consider what happens when we have $n+1$ objects.
This means that each $B^*_1, B^*_2, \cdots, B^*_{n+1}$ must have exactly 1 object inside of them. We clearly see that the statement is true,
as due to the pigeon hole theorem, there must be a hole with at least two objects inside of it. We can then use these two(or more) objects
as proof that we can indeed find two object which are contained in a new box that has fewer objects than the old box that contained them. 
We can then further our argument by adding one object at a time. We can use an algorithm
which takes all objects in the current greatest $B$ and transfers them into the objects in the greatest $B^*$.
Assuming the greatest $B$ is not immediately greater than $B^*$, we get that our
statement is true, when the current greatest $B$ is greater than the current
greatest $B^*$. Now essentially we realize that the least $B^*$ must be greater than the least $B$, something not possible since
it would take $n+1$ new objects to increase the least $B^*$ by $1$. Meanwhile, this would result in a max $B$ with at least 1 greater than before.
Because we can never have that happen, there is always going to be 1 with greater $B$.

\end{document}
