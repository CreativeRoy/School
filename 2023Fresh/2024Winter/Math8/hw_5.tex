\documentclass[12pt]{article}

\usepackage[a4paper, total={6in, 8in}]{geometry}
\usepackage{enumitem}
\usepackage{amssymb}
\usepackage{amsmath}
\usepackage{array}
\usepackage{fancyhdr}
\renewcommand{\headrulewidth}{0pt}

\pagestyle{fancy}

\setlength{\headheight}{35pt} 
\lhead{Math 8 \\ Homework 5} % Left header is empty
\rhead{Roy Zhou \\ \today} % Right header contains date and name

\geometry{margin=1in} 

\begin{document}

\noindent
\textbf{Chapter 6}\\


\noindent
\textbf{Problem 18:}\\

Going by contradiction, we assume that $4|a^2 + b^2$ and we assume that both of $a, b$ are odd. WLOG lets say $a$. 
we end up representing $a = 2n+1$, and $b = 2m+1$ where $n, m \in \mathbb{Z}$. We turn: \\
$$a^2 + b^2 = (2n+1)^2 + (2m+1)^2$$
$$= 4n^2+4n+1 + 4m^2+4m+1$$
$$= 4(n^2 + n + m^2 + m) + 2$$
As we see our expression $4(n^2 + n + m^2 + m) + 2 \equiv 2 \pmod 4$, thus we get a contradiction.\\

\noindent 
\textbf{Chapter 7}\\

\noindent
\textbf{Problem 22:}\\

We can prove this problem through direct proof. Consider the two cases, one in which $n$ is even and one in which n is odd.
In the case in which $n$ is even, we can write $n = 2m$, $m \in \mathbb{Z}$. We see that: 
$$n^2 = 4m^2$$
Which is divisible by 4.
In the case where $n$ is odd we can write $n = 2m+1$, $m \in \mathbb{Z}$. We see that:
$$n^2 - 1 = 4m^2 + 4m = 4(m^2 + m)$$
Again, this is divisible by 4. In both cases, either $n^2-1$ or $n^2$ is divisible by 4, thus we know the statement is true.

\noindent 
\textbf{Problem 24: }\\

To solve this problem we just go by cases, assume $a$ is even we get $a = 2m$ and then 
$$a^2-3 = 4m^2 -3$$
This clearly is a contradiction, as $4m-3 \equiv 1 \pmod 4$.
On the other hand, if $a$ is odd, we get $a = 2m+1$. Again we see:
$$a^2-3 = 4m^2 + 4m - 2$$
This is yet another contradiction, as $4(m^2 + m) - 2 \equiv 2 \pmod 4$. Thus we get that if 
$a \cin \mathbb{Z}$, then $4 \nmid (a^2-3)$.

\noindent 
\textbf{Chapter 8}\\

\noindent
\textbf{Problem 22:}\\

We can prove the problem in the first direction $A \subseteq B \Rightarrow A \cap B = A$
From our given, we get that $\forall x: x \in A \Rightarrow x \in B$ because $x \in B$.
Now we need to prove both that $A \cap B \subseteq A$ and $A \subseteq A \cap B$.
We can do the first by stating the definition of intersection. An intersection by definition
can only be less than or equal to the sets that comprise it. Thus it has to be less than or equal to $A$.
For the second one, we can use our first thing, where $\forall x: x \in A, B$. Which means that 
$x \in A \cap B$. Now we have proved the first direction.
To prove the second direction we do the thing from the opposite direction. Assuming
$A \subseteq B \Leftarrow A \cap B = A$ 
To prove this we need to use our assumption that $A \cap B = A$. $\forall x \in A : x \in B$.
For our proof, this directly translates to our conclusion, which is that $\forall x \in A : x \in B$.
Thus for all $x \in A$ we get that $x \in A \cap B$.\\

\noindent
\textbf{Problem 28:}\\

We can do this problem inductively. Given the base case $a, b = 0$ we get that $12a + 25b = 0$.
We can go up by 1 from every integer by decreasing $a$ by 2 and increasing
$b$ by 1. We can go down by 1 by doing the opposite, increasing $a$ by 2 and decreasing $b$ by 1. 
Now, we have a base case, $12a + 25b = 0$ and we have an inductive step to reach every integer. Thus we are done,
we have proved that $\{12a+25b:a,b\in\mathbb{Z}\}$.

\noindent 
\textbf{Chapter 9}\\



\noindent
\textbf{Problem 10:}\\
The statement given is true, as by definition, subtraction of sets $A-B$ means 
all elements in $A$ that are not in $B$. Because $A \cap B = \emptyset$. This means that $A-B = A$.
So essentially we are getting $\mathcal{P}(A) - \mathcal{P}(B) \subseteq \mathcal{P}(A)$.
This is obviously true, as $\mathcal{P}(B)$ and $\mathcal(P){A}$ only have $\emptyset$ in common.
As a result the result of $\mathcal{P}(A) - \mathcal{P}(B) = \mathcal{P}(A)-\emptyset$. 
This still means, however, that every element of $S = \mathcal{P}(A) - \emptyset$ is inside $\mathcal{P}(A)$.
Thus the statement is true.

\noindent
\textbf{Problem 12:}\\
This problem is obviously not true, we can bring a counter example of $\{a,b,c\} = \{1, 2, 2\}$. 
We get that $\{ab, bc, ac\} = \{2, 4, 2\}$. They are all the same parity, yet $a$ is odd, while $b, c$ are even, so we have a contradiction.\\

\noindent
\textbf{Problem 16:}\\
This is not true again, as we can have $A = \{1, 2, 3, 4, 5 \}$ and $B = \{5, 6, 7, 8\}$.
The sum of the sizes of $|A| + |B| = 9$. But the set $A \cup B = {1, 2, 3, 4, 5, 6, 7, 8}$
and thus $|A \cup B| = 8$. We have a contradiction, so false.

\noindent
\textbf{Problem 22:}\\
This is true, given $p, q$ with $p < q$. This means that $2 \ge q$. Which means that $q$ is always going to be an odd number.
Now we can label $q = 2n+1$. We thus know that $q^2$ is also an odd number, as an odd number squared is odd. Thus we know that $2p + q^2$ is odd
as it represents an even number adding with an odd number, something that creates an odd.

\noindent 
\textbf{Chapter 10}\\

\noindent
\textbf{Problem 2:}\\
We first start off with the base case. Given $n = 1$, we get that $\frac{1(1+1)(3)}{6} = 1^2$. So we do have a base case.
Now for our inductive step. Given that $$\frac{n(n+1)(2n+1)}{6} = 1^2 + 2^2 + \cdots + n^2$$,
we can write:
$$\frac{n(n+1)(2n+1)}{6} + (n+1)^2$$
$$= \frac{n(n+1)(2n+1)}{6} + n^2+2n+1$$
$$= \frac{n(n+1)(2n+1)+6n^2+12n+6}{6}$$
$$= \frac{(n+1)(n+2)(2(n+1)+1)}{6}$$
Thus we have proved the inductive step to be true. Our proof is done, as we have extablished both 
the base case as well as the inductive step.

\noindent
\textbf{Problem 6:}\\
We first start off with the base case. Given $n = 1$, we get
$$(8-5) = 4(1)^2-(1)$$
$$3 = 3$$
Now assume that $\sum^{n}_{i = 1}(8i-5) = 4n^2 - n$. If we were to have $n \rightarrow n-1$ it would look like:
$$\sum^{n}_{i = 1}(8i-5) = 4n^2 - n + (4n + 3)$$
Doing some arithmetic, we get
$= 4n^2 +4n + 4 - (n+1) = 4(n+1)^2 - (n+1)$
Thus we have proved our inductive step and our proof is done.

\end{document}

