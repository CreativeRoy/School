\documentclass[12pt]{article}

\usepackage[a4paper, total={6in, 8in}]{geometry}
\usepackage{enumitem}
\usepackage{amssymb}
\usepackage{amsmath}
\usepackage{array}
\usepackage{fancyhdr}
\renewcommand{\headrulewidth}{0pt}
\pagestyle{fancy}
\geometry{margin=1in}
\geometry{bottom=1.25in}
\setlength{\headheight}{35pt} 

\lhead{Comb 116 \\ Homework 5} % Left header is empty
\rhead{Roy Zhou \\ \today} % Right header contains date and name

\begin{document}

\begin{center}
	\LARGE	
	\textbf{Inclusion-Exclusion Principle Problems}
\end{center}

\noindent
\textbf{Problem 2:}\\

\noindent
We call $P_1$ the property of not divisible by $4$. We call $P_2$ the property of being divisible
by 6, and so on. We have $P_1,P_2,P_3,P_4$, and we are trying to find 
$\overline{P_1} \cap \overline{P_2} \cap \overline{P_3} \cap \overline{P_4}$.
To do this, we use the inclusion-exclusion principle, we write:
\[
\overline{P_1} \cap \overline{P_2} \cap \overline{P_3} \cap \overline{P_4}
= 
|S|-|P_1|-|P_2|-|P_3|-|P_4|+|P_1\cap P_2|+\cdots-|P_2\cap P_3\cap P_4|+|P_1 \cap P_2 \cap P_3 \cap P_4| 
.\] 
Now we just need to calculate each of the terms.

\begin{align}
	\left|P_1\right| = \left\lfloor \frac{10^4}{4}\right\rfloor = 2500\\
	|P_2| = \left\lfloor \frac{10^4}{6}\right\rfloor = 1666\\
	|P_3| = \left\lfloor \frac{10^4}{7}\right\rfloor = 1428\\
	|P_4| = \left\lfloor \frac{10^4}{10}\right\rfloor = 1000\\
	|P_1\cap P_2| = \left\lfloor \frac{10^4}{24} \right\rfloor = 416\\	
	|P_1\cap P_3| = \left\lfloor \frac{10^4}{28} \right\rfloor = 357\\
	|P_1\cap P_4| = \left\lfloor \frac{10^4}{40} \right\rfloor = 250 \\
	|P_2\cap P_3| = \left\lfloor \frac{10^4}{42} \right\rfloor = 238 \\
	|P_2\cap P_4| = \left\lfloor \frac{10^4}{60} \right\rfloor = 166 \\
	|P_3\cap P_4| = \left\lfloor \frac{10^4}{70} \right\rfloor = 142 \\
	|P_1\cap P_2\cap P_3| = \left\lfloor \frac{10^4}{168} \right\rfloor = 59 \\
	|P_1\cap P_2 \cap P_4| = \left\lfloor \frac{10^4}{240} \right\rfloor = 41 \\
	|P_1\cap P_3 \cap P_4| = \left\lfloor \frac{10^4}{280} \right\rfloor = 35 \\
	|P_2\cap P_3 \cap P_4| = \left\lfloor \frac{10^4}{420} \right\rfloor = 23 \\
	|P_1 \cap P_2 \cap P_3 \cap P_4| = \left\lfloor \frac{10^4}{1680} \right\rfloor = 5
\end{align}
Now we can just plug this into our equation, we get:
\[
\overline{P_1} \cap \overline{P_2} \cap \overline{P_3} \cap \overline{P_4} = 
\]
\[
10^{4}-(2500+1666+1428+1000)+(416+357+250+238+166+142)-(59+41+35+23)+5 = 4822
\].\\

\noindent
\textbf{Problem 5:}\\

\noindent
We can use a similar strategy, call $B, C, D$ the properties of having more than 4 $b$'s, 5 $c$'s, and 7 $d$'s, 
respectively. Now we have the equation:
\[
	|\overline{B} \cap \overline{C} \cap \overline{D}| = 
	|S| - (|B|+|C|+|D|) + (|B \cap C| + |B \cap D| + |C \cap D|) - (|B \cap C \cap D|)
.\] 
Again, we calculate the terms.

\begin{align}
	|S| = \binom{10+4-1}{3} = \binom{13}{3}\\
	|B| = \binom{10-5+4-1}{3} = \binom{8}{3}\\
	|C| = \binom{10-6+4-1}{3} = \binom{7}{3}\\
	|D| = \binom{10-8+4-1}{3} = \binom{5}{3}\\
	|B \cap C| = 0\\
	|B \cap D| = 0\\
	|C \cap D| = 0\\
	|B \cap C \cap D| = 0
\end{align}
We get:
\[
	|\overline{B} \cap \overline{C} \cap \overline{D}| = 
	\binom{13}{3}-(\binom{8}{3}+\binom{7}{3}+\binom{5}{3}) = 185	  
.\]\\

\noindent
\textbf{Problem 6:}\\

\noindent
We do the same thing as problems 2 and 5. Call $A,B,C$ the properties of having more than 
6 chocolate, 6 cinnamon, and 3 plain, respectively. We get the equation:
 \[
	|\overline{A} \cap \overline{B} \cap \overline{C}| = 
	|S| - (|A|+|B|+|C|) + (|A \cap B| + |A \cap C| + |B \cap C|) - (|A \cap B \cap C|)
 .\] 

\begin{align}
	|S| = \binom{12+3-1}{2} = \binom{14}{2}\\
	|A| = \binom{12-7+3-1}{2} = \binom{7}{2}\\
	|B| = \binom{12-7+3-1}{2} = \binom{7}{2}\\
	|C| = \binom{12-4+3-1}{2} = \binom{10}{2}\\
	|A \cap B| = 0 \\
	|A \cap C| = \binom{12-7-4+3-1}{2} = \binom{3}{2} = 3\\
	|A \cap B| = \binom{12-7-4+3-1}{2} = \binom{3}{2} = 3\\
 	|A \cap B \cap C| = 0
\end{align}
After plugging in the terms, we get:
\[
|\overline{A} \cap \overline{B} \cap \overline{C}| = 
\binom{14}{2} - (\binom{7}{2} + \binom{7}{2} + \binom{10}{2}) + (3 + 3) = 10
.\]\\
 
\noindent
\textbf{Problem 8}\\

\noindent
We just use the same idea, as the first few problems. We can call $A_i$ the situation when $x_i$ is greater
than 5. We get

\begin{align}
	|S| = \binom{14-5+5-1}{4} = \binom{13}{4} \\
	|A_i| = \binom{14-6-5+5-1}{4} = \binom{8}{4}\\
	|A_i \cap A_j| = 0 \\
	\cdots = 0
\end{align}

Then to put it together, we just write:
\[
	|\overline{A_1} \cap \overline{A_2} \cap\overline{A_3} \cap\overline{A_4} \cap \overline{A_5}| = 
	\binom{13}{4} - 5\binom{8}{4} = 365
.\] 

\noindent
\textbf{Problem 15(a):}\\

\noindent
In the case that no gentleman recieves his own hat, we just directly reference the seventh dearrangement
number, or in notation, $D_7$. We also use the formula, to get that
\[
D_7 = 
7!(1 - \frac{1}{1} + \frac{1}{2!} - \frac{1}{3!} + \frac{1}{4!} - 
\frac{1}{5!} + \frac{1}{6!} - \frac{1}{7!}) = 1854
.\] \\

\noindent
\textbf{Problem 15(b):}\\

\noindent
In the case where at least one gentleman gets his hat, we just do
\[
7! - D_7 = 5040-1854 = 3186
.\] \\

\noindent
\textbf{Problem 15(c):}\\

\noindent
In the case where at least two of the gentleman recieve their own hat, we just get
our previous number, $7!-D_7$, and subtract the scenario in which only one gentleman
recieves his own hat, or $7(D_6)$. We get:
\[
7!-D_7-7D_6 = 2921
.\]\\
 
\noindent
\textbf{Problem 21:}\\

\noindent
To prove the if and only if part, we must prove that $D_n$ is an even number if $n$ is odd,
then we can either prove that if $D_n$ is even $n$ is odd, or we can just prove that $D_n$ 
is odd when $n$ is even. If we do the latter, we have proved the if and only if, as $n$ can
only be either even or odd. If $D_n$ were to be even we know it has to be becase $n$ is odd.
For the first part, we write down the definition of $D_n$ :
\[
D_n = n!(1 - \frac{1}{1!} + \frac{1}{2!} - \cdots + (-1)^{n}\frac{1}{n!})
\] 
\[
	= n! - n! + \frac{n!}{2!} - \frac{n!}{3!} - \cdots + (-1)^{n}\frac{n!}{n!} 
.\] 
Using this abridged form, we are able to draw a few conclusions, because $n-1$ is even, we
know that:
 \[
	 D_n \equiv \frac{n!}{(n-1)!} - \frac{n!}{n!} \pmod{2}
 \]
  \[
	  D_n \equiv n! - n! \equiv 0 \pmod{2}
 .\]
As shown, because $n-1$ is even, we are able to reduce the question down to the terms above.
From the terms above, we get two odds, which implies that their sum is even. Thus we have proven
that if $n$ is odd, that $D_n$ will be even.  \\

\noindent
For the other way, we prove the contrapositive. If $n$ is even, prove that $D_n$ is odd.
This problem works much simpler than the previous one, the same way in which we were able to
cancel out(mod 2) the terms until $\frac{n!}{(n-1)!}$, we can cancel out the terms until
\[
	D_m \equiv \frac{n!}{n!} \pmod{2}
.\] 
Because $\frac{n!}{n!} = 1$, we get tha t
 \[
	 D_m \equiv 1 \pmod{2}
.\] 
Thus we have proved that if $n$ is even, $D_m$ is odd. This directly implies that if $D_m$ is even,
$n$ is odd. Thus we have proved both ways, completing the if-and-only-if proof.
\end{document}
