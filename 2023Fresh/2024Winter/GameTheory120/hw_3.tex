\documentclass[12pt]{article}

\usepackage[a4paper, total={6in, 8in}]{geometry}
\usepackage{enumitem}
\usepackage{amssymb}
\usepackage{amsmath}
\usepackage{array}
\usepackage{fancyhdr}
\renewcommand{\headrulewidth}{0pt}

\pagestyle{fancy}

\setlength{\headheight}{35pt} 
\lhead{Game Theory 120 \\ Homework 3} % Left header is empty
\rhead{Roy Zhou \\ \today} % Right header contains date and name

\geometry{margin=1in} 

\begin{document}

(I haven't yet figured out how to paste images in latex well, so I'll just use a notation
"\textbf{ABCDE}" where every letter is attached to the letter to its left, the leftmost is attatched to the ground)\\

\noindent
\textbf{Problem 1:}\\

Assuming that left is red, and right is blue, we get: 
$$\gamma = \{\textbf{BR}, \bullet \textbf{0} | \textbf{R}\}$$
$$\textbf{BR} = \{\bullet \textbf{0} | \textbf{R}\}$$
$$\textbf{R} = \bullet 1$$

We get that $\textbf{BR} = \bullet \frac{1}{2}$, thus by rules we get $\gamma = 3/4$. \\

\noindent 
\textbf{Problem 2: }\\

Again, using our previous notation where $\bullet \textbf{R} = \bullet 1$,
$\bullet \textbf{B} = \bullet{-1}$, we get that
$$\bullet \textbf{R} = \bullet 1$$
$$\bullet \textbf{BR} = \bullet 1/2$$
$$\bullet \textbf{RBR} = \bullet 3/4$$
$$\bullet \textbf{BRBR} = \{\bullet \textbf{BR}, \bullet 0 | \bullet \textbf{RBR}, 
\bullet \textbf{R}\} = \{\frac{1}{2}, 0 | \frac{3}{4} , 1\} = \bullet \frac{5}{8}$$
$$\bullet \textbf{RBRBR} = \{\textbf{BRBR}, \textbf{BR}, 0 | \textbf{RBR}, \textbf{R}\} 
= \{\frac{5}{8}, \frac{1}{2}, 0 | \frac{3}{4}, 1\} = \bullet \frac{11}{16}$$

Essentially, we get that for every two $RB$:

$$\bullet \textbf{R} = \bullet 1$$
$$\bullet \textbf{RBR} = \bullet \frac{3}{4}$$
$$\bullet \textbf{RBRBR} = \bullet \frac{11}{16}$$
$$\bullet \textbf{RBRBRBR} = \bullet \frac{43}{64}$$

We subtract in order: $\frac{1}{4}, \frac{1}{16}, \frac{1}{64}, \cdots, \frac{1}{4^n}$. 
As a generalization, we get that for odd $S$, where
$S = 2m + 1$. We get $S = 1 - \Sigma^{m}_{i = 1} \frac{1}{4}^{i}$. 
For even $S$, we can write $S = 2n$. We can write this as $S = (2(n-1) + 1) + 1$. 
We can plug $(n-1) = m$ into our odd $S$ expression, then subtract $(\frac{1}{2})(\frac{1}{4})^{n-1}$

\noindent
\textbf{Problem 3: }

Our \textbf{RRL} game of push can end up in a few states. Assuming Richard is on the left, we get: 
$$\textbf{RRL} = \{\textbf{RL\_}, \textbf{\_RL} | \textbf{RL\_} \}$$
$$\textbf{RL\_} = \{\textbf{\_L\_} | \textbf{L\_\_}\}$$
$$\textbf{\_RL} = \{\textbf{R\_L} | \textbf{RL\_}\}$$
$$\textbf{R\_L} = \{\textbf{\_\_L} | \textbf{RL\_}\}$$
What we realize is that no matter what we do, 
in the position $\textbf{RRL}$ Richard will always lose in 3 to 4 moves. 
An equivalent hackenbush game would then be \textbf{B R RB}, with each three being seperate towers. This hackenbush position has a dyadic number of 
$\frac{1}{2}$, thus our answer is $\frac{1}{2}$.
\end{document}

