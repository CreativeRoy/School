\documentclass[12pt]{article}

\usepackage[a4paper, total={6in, 8in}]{geometry}
\usepackage{enumitem}
\usepackage{amsmath}
\usepackage{array}
\usepackage{fancyhdr}
\renewcommand{\headrulewidth}{0pt}

\pagestyle{fancy}

\lhead{Game Theory \\ Homework 4} % Left header is empty
\rhead{Roy Zhou \\ \today} % Right header contains date and name

\geometry{margin=1in}
\noindent 

\begin{document}


\noindent 
\textbf{Problem 1:} \\
For the following matrix to have a saddle point, we would need $x<3$,
as this would mean that $x$ would be the biggest in the row and smallest in the column.
\[
\begin{bmatrix}
3 & 6 \\
x & 0 \\
\end{bmatrix}
\]

\noindent
\textbf{Problem 2a: }\\

Think of the game from Colin's perspective, the matrix becomes: 
\[
\begin{bmatrix}
3 & 1 & -2 & -3 \\
-5 & -3 & 2 & 6 \\
\end{bmatrix}
\]
Let's say Colin has probability $p$ of picking row 1, and $1-p$ of picking row 2.
We get:
 \[
\begin{bmatrix}
	p & 1-p 	
\end{bmatrix}
.\] 
Multiplying the two matrices, we get:
\[
\begin{bmatrix}
	8p-5 & 4p-3 & -4p+2 & -9p+6
\end{bmatrix}
.\] 

Plugging into a graphing calculator, we get that the minimax is at $p = 0.625$, and the 
value of the game is -0.5(from colins perspective). This means that colin should be choosing row 1 
$\frac{0.625}{1}$ percent of the time. This also means that in the totality of the game, Rose will be up
$0.5$ dollars. Solving for Rose, we also get that rose should choose the middle two in half-half way.

\\

\noindent 
\textbf{Problem 2b :} \\
\[
\begin{bmatrix}
-4 & 2 & 0 & 3 & -2 \\
4 & 1 & 0 & -3 & 1\\
\end{bmatrix}
\]
Lets say Rose has probability $p$ of picking row 1, and $1-p$ of picking row 2. We could write:
 \[
\begin{bmatrix}
	p & 1-p 	
\end{bmatrix}
.\] 
Multiplying Rose's matrix with the game matrix we get: 
\[
\begin{bmatrix}
	
\end{bmatrix}
.\] 
Thus we can plug these into a graphing calculator, and find the maximum of the minimum points. 
We find the maximin occurs at one locations, at $x = 0.444$. The expected outcome 
at the locations is $-\frac{1}{3}$. Thus we get that if the game were to last for many rounds, Colin would make 
$\frac{1}{3}$ of a dollar. Rose's most optimal strategy is for $p = 0.444$. Solving for Colin, we get that 
he should choose the last two columns at odds of $\frac{1}{3}$ and $\frac{2}{3}$. 

\noindent
\textbf{Problem 3: } \\

Lets say Colin plays 
 \[
\begin{bmatrix}
	p_1 \\
	p_2 \\ 
	1- p_1 - p_2
\end{bmatrix}
.\] 
We get that the game is 
\[
\begin{bmatrix}
	p_2(a+b)+p_1b-b \\
	-p_1(a+c)-p_2c+c
	p_1b-p_2c
\end{bmatrix}
.\] 
Since we know that there are saddle points at solution sets, we set rows equal to each other to solve:
Setting row 1 to row 3:
$$p_2(a+b)+p_1b-b = p_1b-p_2c$$
$$p_2 = \frac{b}{a}+b+c$$
Setting row 2 to row 3:
$$p_1(a+c)-p_2+c=p_1b-p_2c$$
$$p_1 = \frac{c}{a+b+c}$$
We get that: 
$$1-p_1-p_2 = \frac{a}{a+b+c}$$

We plug these values in, and we multiply by the game to get: 
\[
\begin{bmatrix}
	0 \\
	0 \\
	0
\end{bmatrix}
.\] 
Thus we get that the value of the game is 0.

\end{document}


