\documentclass[12pt]{article}

\usepackage[a4paper, total={6in, 8in}]{geometry}
\usepackage{enumitem}
\usepackage{amssymb}
\usepackage{amsmath}
\usepackage{array}
\usepackage{fancyhdr}
\renewcommand{\headrulewidth}{0pt}
\pagestyle{fancy}
\geometry{margin=1in} 
\setlength{\headheight}{35pt} 

\lhead{Game Theory 120 \\ Homework 5} % Left header is empty
\rhead{Roy Zhou \\ \today} % Right header contains date and name

\begin{document}
\noindent
\textbf{Problem 1:}\\
We first make a tree. We start off with 9 branches, each with representing a unique
set of draws. Then for each branch, we have two other branches of Colins's options of folding
or calling. Then in the calling branch we have two other branches of rose folding or calling.
An example would be something like Colin has Jack, and Rose has Queen. If colin folds, the payoffs 
are (-1, 1), if he calls, we go to Rose's options, if she calls it's (-3, 3), otherwise if she
folds it's (1, -1). The whole tree is a bit tough to draw on latex, so I don't really have a good way to include it.
After that, we can split the matrix into a total of 8 strategies of calling: Always, JQ, QK, JK, Q, K, J. Writing this down in a matrix, we get 
\[
\begin{array}{c|cccccccc}
 & A & QJ & KQ & KJ & J & Q & K & N \\
\hline
A & 0 & 1 & \frac{-1}{3} & \frac{1}{3} & \frac{4}{3} & \frac{2}{3} & 0 & 1 \\
QJ & -1 	& \frac{1}{9} &\frac{-8}{9} &\frac{-2}{9} &\frac{8}{9} &\frac{2}{9} &\frac{-1}{9} & 1\\
KQ & \frac{1}{3} & \frac{10}{9}& \frac{1}{9}& \frac{4}{9}& \frac{11}{9}& \frac{8}{9}& \frac{2}{9}&1 \\
KJ & \frac{-1}{3}& \frac{4}{9}& \frac{-2}{9}& \frac{1}{9}& \frac{8}{9}& \frac{5}{9}& \frac{2}{9}&1 \\
J & \frac{-4}{3}& \frac{-4}{9}& \frac{-7}{9}& \frac{-4}{9}& \frac{4}{9}& \frac{1}{9}& \frac{1}{9}&1 \\
Q & \frac{-2}{3}& \frac{2}{9}& \frac{-4}{9}& \frac{-1}{9}& \frac{7}{9}& \frac{4}{9}& \frac{1}{9}&1 \\
K & 0 & \frac{5}{9}& \frac{2}{9}& \frac{2}{9}& \frac{7}{9}& \frac{7}{9}& \frac{4}{9}&1 \\
N & -1 & \frac{-1}{3}& \frac{-1}{3}& \frac{-1}{3}& \frac{1}{3}& \frac{1}{3}& \frac{1}{3}&1 \\
\end{array}
\]
We are able to reduce until we get 
\[
\begin{array}{c|cc}
      & A & KQ \\
\hline
    KQ & \frac{1}{3} & \frac{1}{9} \\
    K & 0 & \frac{2}{9} \\
\end{array}
\]
From here we can just do the trick to find the vectors for colin and rose.

\begin{align}
	|\frac{1}{3} - \frac{1}{9}| = \frac{2}{9}\\
	|0 - \frac{2}{9}| = \frac{2}{9}\\
\end{align}
We get that Rose plays KQ, and K at half half percentages. In other words, Rose would 
always call with K, and half chance to call with Q. For colin, we get:
\begin{align}
	|\frac{1}{3} - 0| = \frac{1}{3}\\
	|\frac{1}{9} - \frac{2}{9}| = \frac{1}{9}  
\end{align}
We swap, then get that colin plays All $\frac{1}{4}$ of the time, and KQ at $\frac{3}{4}$ of the time. 
Again, that means that colin always calls with Q or K, and calls with J a fourth of the time.
This all leads to an expected payoff of $\frac{1}{6}$ for Rose(aka, the second player).\\
\noindent
\textbf{Problem 2: } \\
\textbf{a)} \\
If $y \le 2$ we can eliminate the second row and then the game beccomes a matter of 
whether  $x > 0$. If it is, the game will end up on  $(4,x)$ otherwise, it will be $(2,0)$.
If $x \le 0$ We can eliminate the first column and then the game becomes whether $y > 2$. If it is,
we land on the second row, otherwise, the first. Thus the only ambiguous game is when
$x > 0, y > 2$. In this scenario, we do the trick, getting:
\begin{align}
	|x-0| = x \\
	|2-3| = 1 \\
	\begin{bmatrix}
		\frac{1}{x+1} &
		\frac{x}{x+1}
	\end{bmatrix}\\
	|4-1| = 3 \\
	|2 - y| = y-2 \\
	\begin{bmatrix}
		\frac{y-2}{y+1}\\
		\frac{3}{y+1}
	\end{bmatrix}
\end{align}\\

\textbf{b) }\\
If $y \ge 5$, we eliminate the first row, then we eliminate the first column, since $0 < 1$.
If $y \l 5$, then it comes down to whether $x \le -1$ if it is, we eliminate the first column
, then eliminate the first row because  $-1 < 1$. Thus the only ambigious gamestate is when
 $x > -1, y < 5$. We do the trick to get: 
\begin{align}
 |x-(-1)| = x+1 \\
 |0 - 1| = -1 \\
 \begin{bmatrix}
	 \frac{1}{x+2} & 
	 \frac{x+1}{x+2}
 \end{bmatrix}\\
 |5-y| = 5-y \\
 |1-3| = 2 \\
 \begin{bmatrix}
 	\frac{2}{7-y}\\
 	\frac{5-y}{7-y}
 \end{bmatrix}
\end{align}\\

\textbf{c) } \\
If $y \le 3$, we eliminate the first column. It then becomes a question of if $z \le 1$. If it is, we choose the first row,
otherwise the second. We now split it up into 3 cases, where $x > 0, x = 0, x < 0$:\\

\noindent
\textbf{$x > 0, y > 3, z > 1$}\\
Using the trick, we get: 
\[
\begin{bmatrix}
	\frac{6}{y+3} & \frac{y-3}{y+3}
\end{bmatrix}
\begin{bmatrix}
	\frac{z-1}{z+x-1} \\
	\frac{x}{z+x-1}
\end{bmatrix}
\] \\
\textbf{$x > 0, y > 3, z = 1$} \\
Remove second row, we land on $(x,y)$\\
\textbf{$x > 0, y > 3, z < 1$} \\
Remove second row, we land on $(x,y)$\\
\textbf{$x = 0, y > 3, z > 1$} \\
Remove first row, we land on $(z,6)$ \\
\textbf{$x = 0, y > 3, z = 1$} \\
Using the trick we get:
\[
\begin{bmatrix}
	\frac{6}{y+3} & \frac{y-3}{y+3}
\end{bmatrix}
.\] It doesn't matter what Rose chooses, as her payoffs are identical.\\
\textbf{$x = 0, y > 3, z < 1$} \\
Remove second row, we land on $(x,y)$ \\
\textbf{$x < 0, y > 3, z > 1$} \\
Remove first row, we land on $(z,6)$ \\
\textbf{$x < 0, y > 3, z = 1$} \\
Remove first row, we land on $(z,6)$ \\
\textbf{$x < 0, y > 3, z < 1$} \\
Using the trick, we get: 
\[
\begin{bmatrix}
	\frac{6}{y+3} & \frac{y-3}{y+3}
\end{bmatrix}
\begin{bmatrix}
	\frac{1-z}{z+x-1} \\
	\frac{-x}{z+x-1}
\end{bmatrix}

.\] 
 \noindent
\textbf{Problem 3: }\\
If $p^{T} \equiv Rq$, and there are no pure nash equilibria, then we can assume that the 
row of all 1's is in the rowspace of C,
and the column of all 1's is in the columnspace of R. Thus we can take an example of (R,C), 
such as:
\[
\begin{bmatrix}
	(-1, 2) & (3, 3) & (2, 5)
	(5, 3) & (2, 7) & (4, 3)
\end{bmatrix}
.\]
After drawing the arrows, we realize this isn't a pure nash equilibrium. Now we get that
$C^{T}p = p^{T}C = [1, 1, 1]^{T}$ must be true. However, after bringing the augmented
matrix to reduced row echelon form, we get:
\[
\begin{bmatrix}
	1 & 0 & |-\frac{4}{5} \\
	0 & 1 & |-\frac{1}{5} \\
	0 & 0 & |-\frac{12}{5}
\end{bmatrix}
.\] This makes it so that there is not p such that $p^{T}C = [1, 1, 1]^{T}$ is true. Thus
this is a counterexample, as the row of all 1's isn't in the rowspace, meaning they are not all
equal.

\noindent
 \textbf{Problem 4: } \\
Just to simplify the matrix, we can just say that the batting team gets a run when
the batter hits, while the other team prevents a run. Then we can see it as a 
zero-sum game. 
We get this matrix: \\
\[
\begin{bmatrix}
	0.255 & 0.275 \\
	0.275 & 0.265 \\
\end{bmatrix} \\
\]
\noindent
Now we can quickly find the percentage by doing
\begin{align}
	|0.255-0.275| = 0.02 \\
	|0.275-0.265| = 0.01
\end{align}
We then swap their positions and get that the batting team has a matrix of
$\begin{bmatrix}
	\frac{1}{3} \\
	\frac{2}{3}
\end{bmatrix}.$
On the other hand, for the pitches, we do the same, but for columns: 
\begin{align}
	|0.255-0.275| = 0.02 \\
	|0.275-0.265| = 0.01
\end{align}
We swap, then get a matrix of $
\begin{bmatrix}
	\frac{1}{3} & \frac{2}{3}
\end{bmatrix}.$
So we get that both teams want a third of their players lefty and a two-thirds righty.
This matches somewhat with reality, as in 2019 $\frac{2}{3}$ of the pitchers were right handed.

\end{document}
