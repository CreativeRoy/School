\documentclass[12pt]{article}

\usepackage[a4paper, total={6in, 8in}]{geometry}
\usepackage{enumitem}
\usepackage{amssymb}
\usepackage{amsmath}
\usepackage{array}
\usepackage{fancyhdr}
\renewcommand{\headrulewidth}{0pt}
\pagestyle{fancy}
\geometry{margin=1in} 
\setlength{\headheight}{35pt} 

\lhead{Game Theory \\ Homework 6} % Left header is empty
\rhead{Roy Zhou \\ \today} % Right header contains date and name

\begin{document}
\noindent
\textbf{Problem 1(a): }\\
We can model the Pareto optimal set, $P$, as $x+y=k$, where $k$ is the y-intercept of the $P$, and 
the SQ is at the origin. Now our goal is to maximize the value of $N$, where  $N = (x-x_0)(y-y_0) = xy$. 
We can use a bit of calculus to do this, first substitution $y$ to get
\[
N = x(k-x) = -x^2+kx
.\] 
From there we can take the derivative of $N$, then find the value of $x$ when $N' = 0$.
\begin{align*}
	N' = -2x+k \\
	0 = -2x+k \\
	x = \frac{k}{2}		
\end{align*}
Now since we know that $N$ is greatest(we don't really care about endpoints in the scope of this problem)
when $x = \frac{k}{2}$, $y = \frac{k}{2}$. So we get that line with slope 
$\frac{\frac{k}{2}-0}{\frac{k}{2}-0} = 1$ is the line that produces the solution of the Nash arbitration scheme. \\

\noindent
\textbf{Problem 1(b): }\\
If $-1$ is replaced by $-m$ we can use the same approach just with modifying the equation a bit. We write
$mx+y=k$, and we run the same operation, getting $N = xy = x(k-mx)$. We then get
\begin{align*}
	N = xk-mx^2 \\
	N' = k-2mx \\
	0 = k-2mx \\
	x = \frac{k}{2m}
\end{align*}
Now we find that when $x = \frac{k}{2m}$, $y = \frac{k}{2}$. So we get that the slope is 
$\frac{\frac{k}{2}}{\frac{k}{2m}} = m$. Thus the line that produces the solution to the Nash arbitration scheme 
is just $m$. \\

\noindent
\textbf{Problem 1(c): } \\
Graphing this problem out, we get that the Pareto optimal set is just the segment from $(3,7)$ to $(6,1)$.
Thus we can simply apply what we learned from the second problem, which is that for a given Pareto optimal set,
a line from SQ with the negative slope of that of the $P$ set will produce the intersection/solution for the 
Nash arbitration scheme. We get that the slope of the Pareto optimal set is
\[
m = \frac{7-1}{3-6} = -2
.\] 
Thus we get that the slope from our SQ, or $(3, 1)$ is $2$. So we write the equation $y = 2x-5$. This line
has an intersection with the Pareto optimal set at, $(\frac{9}{2}, 4)$, thus we know the solution is 
$(\frac{9}{2}, 4)$. \\

\noindent
\textbf{Problem 2: } \\
For this problem I first shifted everything such that the SQ was at the origin, then I 
graphed it on desmos (I still haven't figured out a good way to draw on vimtex yet)\\ 
https://www.desmos.com/calculator/njw7y4mkb3 \\
From the graph, we are able to see that our strategy from Problem 1 doesn't work anymore, 
as none of the segments of our Pareto Optimal set have intersections. This is ok, however, becuase as in calculus, 
if there isn't a region where the derivative is equal to zero, we can simply check all of the endpoints. 
So we do that, and we find that the point $(16,17)$ results in $272$, the greatest value, which makes $(16,17)$ the solution to the Nash arbitration
scheme.  \\

\noindent
\textbf{Problem 3: } \\
We first calculate that:
\begin{align*}
	S^{+}V^{+}B^{+} = (9, -9,)\\
	S^{+}V^{+}B^{-} = (7,-8)\\
	S^{+}V^{-}B^{+} = (6, -7)\\
	S^{+}V^{-}B^{-} = (4,-6)\\
	S^{-}V^{+}B^{+} = (1, 3)\\
	S^{-}V^{+}B^{-} = (-1, 4)\\
	S^{-}V^{-}B^{+} = (-2, 5)\\
	S^{-}V^{-}B^{-} = (-4, 6)\\
\end{align*}
Again, I graphed this on desmos: \\
https://www.desmos.com/calculator/1bupc8oym8 \\
We have two possible solutions, one at the intersection of the slope, using our method from problem 1, 
$(\frac{3}{2}, \frac{9}{4})$, and another at the endpoint $(1,3)$ now to check which one is the most optimal,
we simply multiply the x and y, and compare which value is bigger. We get that $(\frac{3}{2}, \frac{9}{4})$ is
the most optimal, as its value is $\frac{27}{8}$, which is larger than the value of $3$. Now because our solution
is between $S^{-}V^{+}B^{+}$ and $S^{+}V^{+}B^{+}$, we get that the vacation bonus and signing bonus are garunteed,
we just need to calculate the salary. We get
\[
75000+3200(\frac{\frac{3}{2}-1}{9-1}) = 75000+3200\frac{1}{16}=75000+200=75200
.\] 
\end{document}	
