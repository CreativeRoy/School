\documentclass[12pt]{article}

\usepackage[a4paper, total={6in, 8in}]{geometry}
\usepackage{enumitem}
\usepackage{amsmath}
\usepackage{array}
\usepackage{fancyhdr}
\renewcommand{\headrulewidth}{0pt}

\pagestyle{fancy}

\lhead{} % Left header is empty
\rhead{Roy Zhou \\ \today \\ Math 8} % Right header contains date and name

\geometry{margin=1in}
\noindent 

\begin{document}
\section*{2.3}

\subsection*{Exc. 2}
$P = $ the function is differentiable. \\
$Q = $ the function is continuous. \\
If a function is differentiable, then the function is continuous.

\subsection*{Exc. 8}
$P = |r| < 1$ \\ 
$Q = $ a geometric series with ratio r converges \\
If $ |r| < 1$ then a geometric series with ratio $r$ converges.

\section*{2.4}

\subsection*{Exc. 2}
$P = $ a function has a constant derivative \\
$Q = $ a function is linear \\ 
A function has a constant derivative if and only if it is linear.

\subsection*{Exc. 4}
$P = $ the occurence becomes an adventure \\ 
$Q = $ one recounts the occurence \\ 
An occurence becomes an adventure if and only if one recounts it. 

\section*{2.5}

\subsection*{Exc. 2}

\begin{tabular}{|c|c|c|c|c|}
    \hline
	\(Q\) & \(R\) & \(Q \lor R\) & \(R \land Q\) & \((Q \lor R ) \leftrightarrow (R \land Q)\) \\
    \hline
	T & T & T & T & T\\
	T & F & T & F & F\\
	F & T & T & F & F\\
	F & F & F & F & T\\
    \hline
\end{tabular}

\subsection*{Exc. 4}

\begin{tabular}{|c|c|c|}
    \hline
	\(P\) & \(Q\) & \(\neg(P \lor Q) \lor (\neg P) \)\\
    \hline
	T & T & F\\
	T & F & F\\
	F & T & T\\
	F & F & T\\
    \hline
\end{tabular}

\subsection*{Exc. 6}

\begin{tabular}{|c|c|c|}
    \hline
	\(P\) & \(Q\) & \((P \land \neg P) \land Q \)\\
    \hline
	T & T & F\\
	T & F & F\\
	F & T & F\\
	F & F & F\\
    \hline
\end{tabular}

\subsection*{Exc. 8}

\begin{tabular}{|c|c|c|c|}
    \hline
	\(P\) & \(Q\) & \(R\) & \(P \lor (Q \land \neg R)\)\\
    \hline
	T & T & T & T\\
	T & T & F & T\\
	T & F & T & T\\
	F & T & T & F\\
	F & F & T & F\\
	F & T & F & T\\
	T & F & F & T\\
	F & F & F & F\\
    \hline
\end{tabular}

\section*{2.6}

\subsection*{Exc. 2}

\begin{tabular}{|c|c|c|c|c|}
    \hline
	\(P\) & \(Q\) & \(R\) & \(P \lor (Q \land R)\) & \((P \lor Q) \land (P \lor R)\)\\
    \hline
	T & T & T & T & T\\
	T & T & F & T & T\\
	T & F & T & T & T\\
	F & T & T & T & T\\
	F & F & T & F & F\\
	F & T & F & F & F\\
	T & F & F & T & T\\
	F & F & F & F & F\\
    \hline
\end{tabular}

\subsection*{Exc. 6}

\begin{tabular}{|c|c|c|c|c|}
    \hline
	\(P\) & \(Q\) & \(R\) & \(\neg(P \land Q \land R)\) & \((\neg P) \lor (\neg Q) \lor (\neg R)\)\\
    \hline
	T & T & T & F & F\\
	T & T & F & T & T\\
	T & F & T & T & T\\
	F & T & T & T & T\\
	F & F & T & T & T\\
	F & T & F & T & T\\
	T & F & F & T & T\\
	F & F & F & T & T\\
    \hline
\end{tabular}

\subsection*{Exc. 8}

\begin{tabular}{|c|c|c|c|}
    \hline
	\(P\) & \(Q\) & \( \neg P \leftrightarrow Q \) & \( (P \rightarrow \neg Q) \land (\neg Q \rightarrow P) \) \\
    \hline
	T & T & F & F \\
	T & F & T & T \\
	F & T & T & T \\
	F & F & F & F \\

    \hline
\end{tabular}

\subsection*{Exc. 10}
We simplify the second statement through DeMorgan's law to become: 
$\neg (P \land \neg Q) \lor R$ further: $(\neg P \lor Q) \lor R$. We realize that they are equivalent, as 
$(\neg P \lor Q)$ is logically equivalent to $P \rightarrow Q$.


\subsection*{Exc. 12}
The statements are logically equivalent, both are true when P is true and Q is false,
they are all false otherwise.

\subsection*{Exc. 14}
The statements are logically equivalent. When P is false both statements are false,
but when P is true both statements are true. 

\end{document}
