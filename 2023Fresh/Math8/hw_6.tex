\documentclass[12pt]{article}

\usepackage[a4paper, total={6in, 8in}]{geometry}
\usepackage{enumitem}
\usepackage{amssymb}
\usepackage{amsmath}
\usepackage{array}
\usepackage{fancyhdr}
\renewcommand{\headrulewidth}{0pt}
\pagestyle{fancy}
\geometry{margin=1in} 
\setlength{\headheight}{35pt} 

\lhead{Math 8 \\ Homework 6} % Left header is empty
\rhead{Roy Zhou \\ \today} % Right header contains date and name

\begin{document}
\noindent
\textbf{Chapter 10}\\

\noindent
\textbf{Problem 10}\\
We first test the base case, where $n = 0$. This works, as $3|0$. Now for the inductive
step, have  $3|5^{2n}-1$. Now we wish to prove that: 
\[
	3|5^{2(n+1)}-1
.\] 
First we can represent:
\begin{align}
	5^{2(n+1)}-1 = 5^{2n}-1 + (25-1)(5^{2n})
\end{align}
 The important thing, is that the difference, is $(25-1)(5^{2n})$ which is a factor
 of $3$, as $24$ is a factor of $3$. Thus we have proved through induction. \\
 
\noindent
\textbf{Problem 12}\\
This clearly doesn't work, as when we check the base case, we get $9|4+8$ or
$9|12$. Which is clearly false.  \\

\noindent
\textbf{Problem 22} \\
Assuming that $0$ is not a natural number, the base case is when $n = 1$.
We get that: 
 \[
	 (1-\frac{1}{2}) \ge \frac{1}{2}
.\] 
This is clearly true, thus we move on to the inductive step. Consider the following to be true:
\[
	\prod^{n}_{i=1}(1-\frac{1}{2^i}) \ge \frac{1}{4} + \frac{1}{2^{n+1}} 
\].
Now for the inductive step, to get 
\[
	\prod^{n+1}_{i=1}(1-\frac{1}{2^i}) \ge \frac{1}{4} + \frac{1}{2^{n+2}} 
\]
, we multiply the left side by $1-\frac{1}{2^{n+1}}$, while we 
subtract the right side by $\frac{1}{2^{n+2}}$. Obviously, we can't really compare
multiplication to subtraction, so we can turn the multiplication to addition by saying:
\[
	(1-\frac{1}{2^{n+1}})\prod^{n}_{i=1}(1-\frac{1}{2^i}) =
	\prod^{n}_{i=1}(1-\frac{1}{2^i}) - \frac{1}{2^{n+1}}\prod^{n}_{i=1}(1-\frac{1}{2^i}) 
.\]
Thus is becomes a question of whether $\frac{1}{2^{n+1}}\prod^{n}_{i=1}(1-\frac{1}{2^i})$
or $\frac{1}{2^{n+2}}$ is greater. Canceling out the $\frac{1}{2^{n+1}}$, the question gets 
further reduced to $\prod^{n}_{i=1}(1-\frac{1}{2^i})$ vs $\frac{1}{2}$. Obviously $\frac{1}{2}$
is bigger as $\prod^{n}_{i=1}(1-\frac{1}{2^i})$ is only equal to $\frac{1}{2}$
when $n = 1$. It gets smaller after every iteration. Thus we know that the left side gets subtracted
for less after each time. This helps confirm our inductive step, as we know that the left side starts off 
equal to the right, yet it gets subtracted for less than or equal each time. \\

\noindent
\textbf{Problem 26}\\
We do this with induction, consider the base case when $n=1$ we get that:  \[
F_0F_1 = F_1^2
.\] 
This is true, as $F_0, F_1 = 1$, and $1 = 1$ is true. Now for the inductive step, 
we consider some $n$ such that 
$$\sum_{k=1}^{n}F^2_k = F_nF_{n+1}$$
is true. \\
Now for the inductive step, we're adding $F_{k+1}^2$ to the left side, while multiplying by 
$\frac{F_{n+2}}{F_n}$ on the right side to get:
$$\sum_{k=1}^{n+1}F^2_k = F_{n+1}F_{n+2}$$
Obviously we have to convert the multiplication to addition to compare. We first get:
$$\frac{F_{n+2}}{F_n} = 1 + \frac{F_{n+1}}{F_n}$$
Thus we know that we are adding to the right
\[
	F_nF_{n+1}\frac{F_{n+1}}{F_n} = F_{n+1}^2
.\] Thus we have proved the inductive step, as both sides are being added $F^2_{n+1}$
after every step, thus maintaining their equality.

\noindent
\textbf{Chapter 11.2}\\

\noindent
\textbf{Problem 2}\\
$R$ is not reflexive or symmetric, as $(a,a),(b,a)$ are lacking from set $A$. On the other hand
$R$ is transitive.

\noindent
\textbf{Problem 8}\\
The relation is the equivalence relation, as for $|x-y| < 1$ on the integers, 
$x$ has to equal $y$. This relation is reflexive, symmetric and transitive. \\

\noindent
\textbf{Problem 14}\\
If $R$ is symmetric(also the conditions given in the problem), 
that must mean there exists some $a,b \in A$ such that:
 \[
	 R = \{(a,b), (b,a), \cdots\}
.\] We also know $R$ is transitive, so we get that: 
$$(a, b) \land (b,c) \Rightarrow (a,c)$$. 
Thus with our $(a,b), (b,a)$ we get that $(a,a),(b,b)$ must exist.
Since there is at least one element that has a relation with every other member
this means that we get that all elements have a relation with themselves. This 
is the definition of reflexive.\\
\noindent
\textbf{Problem 16}\\
To prove reflexive property, we have $x^2 \equiv x^2 \pmod{4}$. This is obviously true. 
To prove the symmetric property, we can take $x^2 \equiv y^2 \pmod_4$. We turn this into
$x^2 + 4a = y^2 + 4b$. Using the symmetric property of arithmetic, we get $y^2 + 4b=x^2+4a$.
Then we can reform this into $y^2 \equiv x^2 \pmod4$. Lastly, to prove transitivity, we get 
\begin{align}
	x^2 + 4a = y^2 + 4b \\
	y^2 + 4c = z^2 + 4d
\end{align}
Shifting around, we get 
\begin{align}
	y^2 = z^2+4(d-c) \\
	
	x^2 + 4a = y^2 + 4b \\
	
	x^2 + 4a = z^2 + 4(b+d-c) 
\end{align}
Which means that:
\begin{align}
	x^2 \equiv z^2 \pmod{4}
\end{align}.
Thus we have proved transitivity. 
\end{document}
